\documentclass[a4paper, 12pt, oneside]{book}

\usepackage[margin=1in]{geometry}
\usepackage{hyperref}
\hypersetup{
    colorlinks,
    citecolor=black,
    filecolor=black,
    linkcolor=black,
    bookmarksopen=true,
    pdftitle={Recipes},
    pdfauthor={Dhairya Patel},
    pdfsubject={Recipes},
    pdfkeywords={Recipes},
    pdfcreator={LaTeX},
    pdfpagemode={UseOutlines}
}
\usepackage{subfiles}
\usepackage[english]{babel}
% Write in Gujarati using Babel, import the fonts to do this.
\babelprovide[import]{gujarati}
\babelfont{rm}[
    Path = ./fonts/Noto_Serif/,
    Extension = .ttf,
    UprightFont = *-Regular,
    BoldFont = *-Bold,
    ItalicFont=*-Italic,
    BoldItalicFont=*-BoldItalic]
    {NotoSerif}
\babelfont{sf}[
    Path = ./fonts/Noto_Sans/,
    Extension = .ttf,
    UprightFont = *-Regular,
    BoldFont = *-Bold,
    ItalicFont=*-Italic,
    BoldItalicFont=*-BoldItalic]
    {NotoSans}
\babelfont{tt}[
    Path = ./fonts/Noto_Sans_Mono/static/,
    Extension = .ttf,
    UprightFont = *-Regular,
    BoldFont = *-Bold]
    {NotoSansMono}
\babelfont[gujarati]{rm}[
    Path = ./fonts/Noto_Serif_Gujarati/static/,
    Extension = .ttf,
    UprightFont = *-Regular,
    BoldFont = *-Bold]
    {NotoSerifGujarati}
\babelfont[gujarati]{sf}[
    Path = ./fonts/Noto_Sans_Gujarati/static/,
    Extension = .ttf,
    UprightFont = *-Regular,
    BoldFont = *-Bold]
    {NotoSansGujarati}
\usepackage{titlesec}
% Modify chapter and section formatting to add horizontal rule below chapter title and remove excessive spacing
\titleformat{\chapter}[display]{\normalfont\huge}{\chaptertitlename\ \thechapter}{20pt}{\Huge}[\vspace{20pt}\titlerule]
\titlespacing*{\chapter}{0pt}{-50pt}{20pt}
\titleformat{\section}{\normalfont\Large}{\thesection}{1ex}{}
\titlespacing*{\section}{0pt}{10pt}{10pt}
\usepackage{tocloft}
% Add dots to chapters in table of contents
\renewcommand{\cftchapleader}{\cftdotfill{\cftdotsep}}
% Remove bolding of chapter entries in table of contents
\renewcommand{\cftchapfont}{\normalfont}
\renewcommand{\cftchappagefont}{\normalfont}

% New environment to type Gujarati in
\newenvironment{gujarati}{\begin{otherlanguage}{gujarati}}{\end{otherlanguage}}
% New command to insert English in the above Gujarati environment
\newcommand{\english}[1]{\foreignlanguage{english}{#1}}

% Set up title page
\title{\Huge Recipes}
\author{Dhairya Patel}
\date{April 27, 2023 \textendash\ \today}

\begin{document}

\pagestyle{empty}

\maketitle

\frontmatter

% \chapter{Dedication}

% \chapter{Copyright}

% \chapter{Acknowledgements}

\tableofcontents

\mainmatter

\begin{gujarati}

\setlocalecaption{gujarati}{part}{Part}
\part{ફરસાણ (Salty Savoury Snacks)}

\subfile{individual_recipes/farsan/surti_locho}
\subfile{individual_recipes/farsan/vati_dalna_khaman}

\setlocalecaption{gujarati}{part}{Part}
\part{મીઠાઈ (Sweets)}

\subfile{individual_recipes/sweets/nan_khatai}

\setlocalecaption{gujarati}{part}{Part}
\part{શાક (Shaak)}

\subfile{individual_recipes/shaak/tameta_bataka}
\subfile{individual_recipes/shaak/ringan_bataka}

\setlocalecaption{gujarati}{part}{Part}
\part{કઠોળ (Beans)}

\subfile{individual_recipes/beans/chole_chana}
\subfile{individual_recipes/beans/rajma}
\subfile{individual_recipes/beans/udad}
\subfile{individual_recipes/beans/mixed_bean_soup}

\part{પંજાબી (Punjabi)}

\subfile{individual_recipes/punjabi/paneer}
\subfile{individual_recipes/punjabi/aloo_paratha}

\part{South Indian}

\subfile{individual_recipes/south_indian/idli_idada_uttappa_batter}

\part{Other}

\subfile{individual_recipes/other/banana_loaf_bread}

\end{gujarati}

\end{document}