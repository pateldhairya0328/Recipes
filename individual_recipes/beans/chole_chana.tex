\documentclass[../../recipes.tex]{subfiles}

\begin{document}

\begin{gujarati}

\setlocalecaption{gujarati}{chapter}{Recipe}
\chapter{છોલે ચાના (Chole Chana)}

\section*{Ingredients}

\begin{itemize}
    \item 1 cup છોલે ચાના.
    \item 1 large tomato and 1/2 large onion.
    \item Salt.
    \item Oil.
    \item રાઈ.
    \item જેરુ.
    \item તમાલપત્ર.
    \item હિંગ.
    \item લાલ મરચું.
    \item હળદર.
    \item ધાણાજીરું.
    \item ચણાનો મસાલો.
    \item 1 green chili pepper.
    \item 3 cloves of garlic or garlic powder.
\end{itemize}

\noindent
\section*{Procedure}

\begin{enumerate}
    \item Soak the ચાના over night.
    \item Purée tomatoes and onion together.
    \item Set instant pot to sauté at normal heat, and add oil.
    \item When instant pot shows hot, add રાઈ, તમાલપત્ર, જેરુ, હિંગ. Optionally add a green chili pepper cut in two halves. Then add the tomato-onion purée. Add લાલ મરચું, હળદર, ધાણાજીરું, ચણાનો મસાલો and salt. Add 5 cloves of garlic finely chopped, or garlic powder. Let this simmer for a little bit.
    \item Add ચાના and water. Then close instant pot and pressure cook on high for 40 minutes. When you open pot, crush some of the beans against the wall of the pot and mix them into the gravy to thicken it.
\end{enumerate}

\end{gujarati}

\end{document}