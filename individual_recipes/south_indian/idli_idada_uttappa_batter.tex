\documentclass[../../recipes.tex]{subfiles}

\begin{document}

\begin{gujarati}

\setlocalecaption{gujarati}{chapter}{Recipe}
\chapter{ઈડલી/ઇદડા/ઉત્તપ્પા Batter (Idli/Idada/Uttappa Batter)}

\section*{Ingredients}

\begin{itemize}
    \item 1/2 cup અડદની દળ.
    \item 1/2 cup rice.
    \item 1/2 ક્યુપિ thick પોહા.
    \item 3/4 spoon મેથીના દાણા.
    \item 2 spoons દહીં.
    \item આદુ મરચું (ઇદડા).
    \item Salt (ઇદડા)
    \item Eno (ઇદડા and ઈડલી).
    \item Diced vegetables (ઉત્તપ્પા).
\end{itemize}

\noindent
\section*{Procedure}

\begin{enumerate}
    \item Soak અડદની દળ, rice, thick પોહા and મેથીના દાણા after washing three times, for at least 6-8 hours (soak it in the morning).
    \item Drain out excess water from soaked mixture (in the evening).
    \item Blend together mixture with 2 spoons દહીં.
    \item Mix batter with spatula, and keep it covered overnight. Then the batter is ready. For ઈડલી/ઇદડા/ઉત્તપ્પા specifically:
    \begin{enumerate}
        \item Add આદુ મરચું and salt for ઇદડા.
        \item Add 1/3 spoon eno and warm water to make ઇદડા and ઈડલી fluffy.
        \item Add vegetables and salt for ઉત્તપ્પા.
    \end{enumerate}
    \item Save batter up to a week in the refrigerator.
\end{enumerate}

\end{gujarati}

\end{document}