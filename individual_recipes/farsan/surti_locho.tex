\documentclass[../../recipes.tex]{subfiles}

\begin{document}

\begin{gujarati}

\setlocalecaption{gujarati}{chapter}{Recipe}
\chapter{સુરતી લોચો (Surti Locho)}

\section*{Ingredients}

\begin{itemize}
    \item 1 વાડકી ચણાની દળ.
    \item 1 મુઠી અડદની દળ.
    \item 1 મુઠી ચોખા.
    \item 2 મુઠી પૌવા.
\end{itemize}

\noindent
\section*{Procedure}

\begin{enumerate}
    \item સાથે મીક્સ કરી 4 વાગે પલાળવું.
    \item 5 વાગે કૃશ.
    \item 1/2 ચમચી સોડા અને 1 ચમચી દહીં નાખી ક્રશ કરવું.
    \item ઇદડા કરતા ઢીલું રાખવું.
    \item ઢાંકીને મૂકવું.
    \item સવારે મીઠું, આદુ-મરચું, અને હળદર નાખવું.
    \item વાઘરીયામાં 1/2 વાઘરીયું તેલ ગરમ કરી થોડો ઈનો ... ખીરામાં નાખવો.
    \item સોડા ... ચમચી ગરમ તેલમાં નાખી તેલ ખીરામાં નાખી હંકાવવું.
    \item થાળી મુકવી.
\end{enumerate}

\end{gujarati}

\end{document}