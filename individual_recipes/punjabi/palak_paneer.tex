\documentclass[../../recipes.tex]{subfiles}

\begin{document}

\begin{gujarati}

\setlocalecaption{gujarati}{chapter}{Recipe}
\chapter{પાલક પનીર (Palak Paneer)}

\section*{Ingredients}

\begin{itemize}
    \item 1 pound પાલક.
    \item 4 tomatoes and 2 onions.
    \item Salt.
    \item Oil.
    \item જેરુ.
    \item તમાલપત્ર.
    \item લાલ મરચું.
    \item હળદર.
    \item ધાણાજીરું.
    \item ગરમ મસાલા.
    \item 1 green chili pepper.
    \item 5 cloves of garlic or garlic powder.
    \item Paneer.
\end{itemize}

\noindent
\section*{Procedure}

\begin{enumerate}
    \item Purée tomatoes and onions together.
    \item Set instant pot to sauté at normal heat, and add oil.
    \item When instant pot shows hot, add તમાલપત્ર and જેરુ. Optionally add a green chili pepper cut in two halves. Then add the tomato-onion purée. Add લાલ મરચું, હળદર, ધાણાજીરું, ગરમ મસાલા and salt. Add 5 cloves of garlic finely chopped, or garlic powder. Let this simmer for a little bit.
    \item Add પાલક and 1/2 cup water. Then close instant pot and pressure cook on high for 4 minutes. When you open pot, use an immersion blender to blend to desired consistency (tilt instant pot at an angle to do this more easily).
    \item Mix in the paneer.
\end{enumerate}

\end{gujarati}

\end{document}